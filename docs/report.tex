\documentclass[conference]{IEEEtran}
\usepackage{cite}
\usepackage{amsmath,amssymb,amsfonts}
\usepackage{algorithmic}
\usepackage{graphicx}
\usepackage{textcomp}
\usepackage{xcolor}

\begin{document}

\title{
  Shared Mobility Simulation Report\\
  {\footnotesize \textsuperscript{*}Note: Sub-titles can be placed here}
}

\author{
  \IEEEauthorblockN{Eduardo Passos}
  \IEEEauthorblockA{
    \textit{MSc in Artificial Intelligence} \\
    \textit{University of Porto}\\
    Porto, Portugal \\
    up202205630@up.pt
  }
  \and
  \IEEEauthorblockN{Guilherme Silva}
  \IEEEauthorblockA{
    \textit{MSc in Artificial Intelligence} \\
    \textit{University of Porto}\\
    Porto, Portugal \\
    up202205298@up.pt
  }
  \and
  \IEEEauthorblockN{Valentina Cadime}
  \IEEEauthorblockA{
    \textit{MSc in Artificial Intelligence} \\
    \textit{University of Porto}\\
    Porto, Portugal \\
    up202206262@up.pt
  }
}


\maketitle

\begin{abstract}

\end{abstract}

\begin{IEEEkeywords}

\end{IEEEkeywords}


% TODO: Remove this section before submission
\section*{Useful Rules:}

\textbf{The 3-part rule: Introduction, Development, Conclusion}
\begin{itemize}
  \item Chapter: 3+ sections
  \item Sections: 3+ paragraphs
  \item Paragraphs: 3+ sentences
\end{itemize}

\textbf{The section-separation rule:}
\begin{itemize}
  \item If the content of a section fits in more subsections, the division should yield at least 2 subsections. Single subsections reflect bad content organization.
\end{itemize}

\textbf{Single-paragraph rule:}
\begin{itemize}
  \item Avoid using single paragraphs in the body of a section. A well-designed section will have at least 3 paragraphs (recall the 3-part rule)
  \item \textbf{Exception - When to use single paragraphs?}
  \begin{itemize}
    \item Abstract (if the size is limited, e.g. as in conference papers)
    \item Conclusion (the same as above)
    \item As an introduction to chapters or sections
  \end{itemize}
\end{itemize}

\textbf{Single-sentence-paragraph rule:}
\begin{itemize}
  \item Avoid using single-sentence paragraphs. A well-designed paragraph will have at least 3 sentences (recall the 3-part rule)
  \item \textbf{Exception - When to use single-sentence paragraphs? To introduce:}
  \begin{itemize}
    \item Figures
    \item Tables
    \item Lists
    \item Algorithms
    \item Equations
    \item Other non-textual elements
  \end{itemize}
\end{itemize}

\textbf{Avoid too much empty space:}
\begin{itemize}
  \item Single-word-ending paragraphs
  \item Whenever possible, try and fill the paragraph width in the last line of the paragraph
  \item Whenever possible, try and fill the paragraph width when adding new figures and tables
  \item Preferably, place figures and tables either at the top or the bottom of the page
  \item Graphic elements in the middle of the page usually leave much empty space
\end{itemize}

\textbf{Figure vs. figure}
\begin{itemize}
  \item Generally, elements such as figures, tables, and equations, are whenever referring to a specific element in the text, are written with the first letter capitalized, followed by the number of the element, e.g. "As depicted in Figure 5, …"
  \item When referring to a figure generically, without its number, it is not capitalized, e.g. "The figure below depicts the proposed architecture…", e.g. "…, as depicted in figures 5-7."
  \item The same is valid for tables and equations. Consistency is key!
\end{itemize}

\textbf{State of the art vs. state-of-the-art}
\begin{itemize}
  \item Some compound nouns can also be used as adjectives, in which case they're written with hyphens, e.g. "This chapter presents the state of the art in AI.", e.g., "The state-of-the-art techniques are detailed in the next section.", e.g. "Data will be collected in real time.", e.g. "Real-time mechanisms support data exchange in our system."
\end{itemize}

\textbf{Numeric citations}
\begin{itemize}
  \item The numeric citations (e.g. [3] or [11-18] are generally used in conference papers due to space limitations) (see IEEE paper formats)
  \item Do not start a sentence with a numeric citation. It doesn't make sense!e.g. "[3] presents an empirical study on ML applied to agriculture."e.g. "Silva (2019) presents an empirical study on ML applied to ..." \textbf{(better!)}
  \item Hint: when the numeric citation is deleted, the text should still make sense!
  \begin{itemize}
    \item e.g. "In [3], authors present an empirical study on ML applied to agriculture." (rephrase!)
    \item e.g. "Authors present an empirical study on ML applied to agriculture [3]."
  \end{itemize}
\end{itemize}

\textbf{Formal documents}
\begin{itemize}
  \item A dissertation should be written with proper and adequate language, should be correct and objective, and should avoid bias
  \item Avoid contractions, such as:
  \begin{itemize}
    \item Don't $\rightarrow$ do not
    \item Doesn't $\rightarrow$ does not
    \item That's $\rightarrow$ that is
    \item It's $\rightarrow$ it is
    \item Won't $\rightarrow$ will not
    \item And others
  \end{itemize}
  \item Prefer these constructions instead:
  \begin{itemize}
    \item Like $\rightarrow$ such as (e.g. ``characteristics such as'')
    \item So, \dots $\rightarrow$ Therefore (e.g. ``Therefore, it is plausible to\dots'')
    \item On the other hand, \dots $\rightarrow$ make sure you start with ``On the one hand, \dots . On the other hand, \dots .''
  \end{itemize}
\end{itemize}

\textbf{Work vs. works}

Be careful with countable and uncountable nouns
\begin{itemize}
  \item \textbf{Works} (countable) -- self-contained things, as works of art or structures in engineering (e.g. ``The works by Van Gogh'' or bridges, tunnels, etc.)
  \item \textbf{Work} (uncountable) -- the body of knowledge in science is one only, and every scientist contributes a piece to enrich it
  \begin{itemize}
    \item e.g. ``The pieces of work reported by both Mendes (2024) and Machado (2019) suggest that \dots''
    \item Prefer to use ``Conclusions and future work'' instead
  \end{itemize}
  \item Countable alternatives to ``work'':
  \begin{itemize}
    \item ``Carvalho and colleagues carried out a series of studies to demonstrate their hypothesis [5].''
    \item ``Different efforts are reported in the literature on this subject [2, 7, 9-11].''
    \item ``Some empirical experiments were performed, which corroborate that idea [18, 21].''
    \item ``Evaluation methods are also proposed by other authors (Carvalho, 2017; Mendes et al, 2025).''
  \end{itemize}
  \item In academic writing, ``research'' and ``data'' are two uncountable nouns that are notoriously difficult to use correctly. Never add ``s'' to pluralize ``research'' or ``data''. (Note that the word ``researches'' is only correct when used as the third-person singular of the verb ``to research''.)
\end{itemize}

\textbf{Ethical issues:}
\begin{itemize}
  \item Plagiarism and self-plagiarism
  \begin{itemize}
    \item Plagiarism often involves using someone else's words or ideas without proper citation, but you can also plagiarize yourself. Self-plagiarism means reusing work that you have already published or submitted for a class
  \end{itemize}
  \item Figure, data, and other elements
  \begin{itemize}
    \item The use of someone else's figures and art creation implies proper permissions to be used ``as is''
    \item Citation is not enough!
    \begin{itemize}
      \item ``Taken from Author (year)'' or ``borrowed from Author (year) $\rightarrow$ requires authorization
      \item ``Adapted from Author (year)'' $\rightarrow$ it is your own interpretation and requires no authorization
      \item Alternatively, you can include a ``Source:\dots'' field in the Figure caption
    \end{itemize}
  \end{itemize}
\end{itemize}

\clearpage

\section{Introduction}
% TODO: REMOVE THIS WHEN WE CITE REAL REFERENCES - THIS CITATION IS NEEDED TO COMPILE THE DOCUMENT
\textbf{\texttt{[PLACEHOLDER]}} Current shared mobility systems face significant challenges in balancing demand and supply. This report explores these challenges and proposes simulation-based solutions \cite{exampleRef}.

\textit{
  \textbf{TODOs:}
  \begin{itemize}
    \item Context
    \item Problem statement. Clearly define the research question or technical challenge with a precise problem description and formal representations
    \item Motivation (to solve the problem) / significance
    \item Aim and goals
    \item Aim: the main expected outcome of this research project (the big picture!)
    \item Goals: specific objectives to be accomplished
    \item Research questions
    \item Hypotheses
    \item Document/paper structure
  \end{itemize}
}

This project focuses on the daily and strategic management of electric scooter fleets, aiming to minimize operating costs while maximizing the level of service to users, in both quantity and quality. This includes planning redistribution by truck, charging/recharging policies, designing user incentives to reduce imbalances, and demand-prediction algorithms. Practical applications include shared system administrators and researchers, mobility authorities, and smart cities.

The simulation aims to replicate the operation of a shared micromobility system throughout the day, considering variables such as demand patterns, vehicle availability, battery levels, and recharging requirements. It will represent not only users requesting and completing trips, resulting in the movement of shared mobility vehicles throughout the city, but also on-site operators overseeing the shared mobility vehicles and respective stations, and performing fleet relocations to guarantee its most efficient distribution throughout the city without compromising the users' experience.

By simulating the interactions between these entities, the objective is to evaluate and compare different balancing strategies under realistic urban conditions. The simulation will make it possible to measure how each approach affects the service through Key Performance Indicators (KPIs), in other words, measures and benchmarks, as will be described later. Examples include quality (vehicle availability and waiting time), operational efficiency (costs and distance traveled during redistribution), and system sustainability (energy consumption and battery utilization). Furthermore, another area of exploration could be to study the effect of limited battery autonomy and charging times on overall fleet performance.

Ultimately, the simulation provides a controlled and flexible environment to test algorithms and management policies. It allows for the resolution of several problems, such as identification of bottlenecks, the quantification of trade-offs between cost and service quality, and the exploration of optimal or near-optimal strategies to maintain a balanced and efficient shared mobility system.

The goal of the simulation problem is to identify which actions and strategies maximize fleet utilization, improve service quality, and minimize operational inefficiency. This will be done through modeling and analyzing the dynamics of a shared electric vehicle fleet within an urban environment in a certain spatial and temporal context. The focus will be on studying the imbalance in vehicle distribution caused by uneven and time-dependent demand: some areas experience shortages of available vehicles, while others accumulate unused ones.

The eventual real-world manifestation of this process would be to support decision-making in managing shared electric vehicle fleets by evaluating different operational strategies and their impact on system performance. As such, and to pursue the defined goal, the model aims to simulate the system under varying researcher-defined strategies, such as vehicle relocation and charging policies, and to measure KPIs including unmet demand, vehicle availability, energy consumption, and operational costs.


\section{Related Work}

\textit{
  \textbf{TODOs:}
  \begin{itemize}
    \item Review of existing literature.
    \item Relevant concepts and studies.
    \item Summarize background and related work to highlight gaps addressed by the project.
    \item Discussion and critical analysis.
  \end{itemize}
}

The problem falls within the field of Urban Mobility and, more specifically, the subfield of Shared Micromobility and Fleet Management. It addresses how cities can efficiently manage and balance fleets of shared electric vehicles to meet fluctuating demand patterns throughout the day. As a result, and similarly to many of the modelling projects that seek to study the real world, this issue lies at the intersection of several disciplines.

From the perspective of transportation science, the problem concerns the modeling of human movement patterns and travel demand within urban environments. It involves understanding how people choose modes of transport, how their choices change with time of day, weather, or location, and how these behaviors influence the spatial distribution of shared vehicles.

In the area of optimization and operations research, the challenge translates into a set of complex decision problems. These include vehicle rebalancing (determining how and when to move vehicles from low-demand to high-demand areas), routing and scheduling of redistribution trucks, and the design of user incentives to encourage self-balancing behavior. Such problems are often formulated as variants of the Vehicle Routing Problem (VRP), capacitated network flow models, or stochastic optimization problems. Furthermore, the possibility of incorporating machine learning brings a predictive and adaptive layer to the system.


\section{Materials and Methods}

\textit{
  \textbf{TODOs:}
  \begin{itemize}
    \item Problem formalization: what to solve?
    \item Materials:
    \begin{itemize}
      \item Data
      \item Tools
      \item Techniques
    \end{itemize}
    \item Methods: how to solve it?
    \item Solution design
  \end{itemize}
}


\subsection{System Model}
The simulation model comprises several interactions between entities:
\begin{enumerate}
  \item \textbf{Station}: Use to hold vehicles and act as a charging infrastructure. Properties include capacity (slots), number of chargers, and distance/travel time to other stations.
  \item \textbf{Vehicle}: Moves between stations and consumes/charges energy. Tracked by its Battery State of Charge (SoC).
  \item \textbf{Customers}: Represent demand. They are characterized by arrival rates (people per time interval) and origin-destination probabilities.
  \item \textbf{Environment}: External factors affecting the system, such as diurnal variation (time of day), weather factors, and special events (e.g., concerts).
  \item \textbf{Operator}: Represents the control plan using policies for relocation, charging, and special user incentives.
\end{enumerate}

\subsection{System Variables}
The system defines \textbf{Exogenous Variables} which are not controlled by the operator, including demand rate (Poisson arrival intensity), origin-destination probability matrix, weather influence, and travel times.

\textbf{Endogenous Variables} describe the system's internal state, such as the number of vehicles available at each station, average battery SoC, number of waiting customers, and ongoing trips.

The performance is measured by \textbf{Output Variables (KPIs)}, as detailed in Table \ref{tab:kpis}.

\begin{table}[htbp]
  \caption{Key Performance Indicators}
  \begin{center}
    \begin{tabular}{|l|p{3.5cm}|l|}
      \hline
      \textbf{KPI} & \textbf{Meaning} & \textbf{Goal} \\
      \hline
      Availability & Fraction of stations with at least one vehicle & Maximize \\
      \hline
      Unmet demand & Number of users who could not find a scooter & Minimize \\
      \hline
      Standard deviation & Measure of imbalance across the network & Minimize \\
      \hline
      Scooters in use & Vehicles currently rented or moving & Maximize \\
      \hline
      Idle station time & Time periods when a station has no activity & Minimize \\
      \hline
      Relocation distance & Total distance traveled during rebalancing (km) & Minimize \\
      \hline
      Energy cost & Cumulative electricity cost for charging & Minimize \\
      \hline
      Success rate & Ratio of served users to total demand & Maximize \\
      \hline
    \end{tabular}
  \label{tab:kpis}
  \end{center}
\end{table}

\subsection{Decision Criteria}
System performance is evaluated through a multi-objective score that combines Service quality, Operational cost, and Workload. Formally, policies are benchmarked using:
\begin{equation}
  Score = \alpha \cdot Q_{uality} + \beta \cdot C_{ost} + \gamma \cdot W_{orkload}
\end{equation}
where $\alpha$, $\beta$, and $\gamma$ are policy weights.

Baseline policies include:
\begin{itemize}
  \item \textbf{Greedy Rebalancing}: Moving scooters from overfilled stations ($x > 0.8C$) to depleted ones ($x < 0.2C$).
  \item \textbf{Nightly Uniform Rebalancing}: Resetting stations to a uniform fill level ($\approx 60\%$) at 02:00.
  \item \textbf{Random Demand Scenario}: A neutral benchmark without relocation.
\end{itemize}

\subsection{Data Requirements}
The project relies on synthetic and open geographical data to represent a small-scale micromobility system in an urban environment. We use OpenStreetMap (OSM) via the OSMnx library to extract the road/bike network for the city of Porto. A synthetic generation of 15 representative micromobility stations is used, with distances computed from the OSM graph.

Assumptions for the experimental setup include:
\begin{itemize}
  \item 15 stations approximating a city-center zone (1-3 km between nodes).
  \item Constant average speed of 15 km/h.
  \item Customer arrivals follow Poisson processes with mean rate $\lambda = 0.8$ per 5 minutes ($\sim 9.6$ per hour).
  \item Each station has 12 parking slots.
  \item Vehicles are electric scooters with 0.5 kWh capacity and linear charging.
\end{itemize}

\subsection{Tools}
The simulation is implemented in Python. \textbf{NumPy} is used for discrete-time system simulation. \textbf{Matplotlib/Seaborn} are used for KPI plots and time-series analysis. \textbf{NetworkX/OSMnx} are used to build and analyze the station network graph.


\section{Results and Discussion}

\textit{
  \textbf{TODOs:}
  \begin{itemize}
    \item Methods to collect results
    \item Result presentations (graphs, charts, diagrams, tables,
    associations)
    \item Critical discussion of bad results
    \item Critical discussion of good results
    \item Focus on counter-intuitive results
    \item Focus on additional results (other than the rest of the literature)
  \end{itemize}
}

\section{Conclusion and Future Work}
\textit{
  \textbf{TODOs:}
  \begin{itemize}
    \item Remark the conclusions drawn from the related work and gap analysis.
    \item Remark problem and goals.
    \item Remark the main results and findings.
    \item Summarize the main contributions.
    \item Scientific
    \item Application
    \item Technological
    \item Future work
    \item Further developments (how to improve the current work)
    \item Future opportunities / R\&D paths (spin-off projects/problems of the current work)
  \end{itemize}
}

\bibliographystyle{IEEEtran}
\bibliography{references}

\end{document}
